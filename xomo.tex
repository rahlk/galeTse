
\subsection{XOMO: Software Process Models}\label{sec:xomoIs}

The XOMO model~\cite{me07f,me09a,me09e} combines four software process models from Boehm's group at the University of Southern California.
It reports four objective scores (which we will try to minimize): project {\em risk}; development {\em effort} and {\em defects}; and total {\em months} of development.


In this study, optimizers tune the XOMO  decision variables of \fig{cocont}
to improve the following objectives:
\bi
\item Reduce risk;  
\item Reduce effort;
\item Reduce defects;
\item Reduce months.
\ei
Full details of XOMO have been offered in prior papers~\cite{me07f,me09a,me09e}. A summary is offered below.

XOMO uses the variables of  \fig{cocont}  in a variety of models. The XOMO {\em effort} model predicts for 
``development months'' where one month
is 152 work hours by one developer (and includes development and management hours): 
\begin{equation}\label{eq:cocII}
\mathit{effort}=a\prod_i EM_i *\mathit{KLOC}^{b+0.01\sum_j SF_j}
\end{equation}
Here, {\em EM,SF} denote the effort multipliers and scale
factors and
 $a,b$ are the {\em local calibration} parameters which in COCOMO-II
have default values of 2.94 and 0.91.

The variables of \fig{cocont} are also  used in
the COQUALMO defect prediction
model~\cite{boehm00b}.  COQUALMO assumes that
certain variable settings {\em add} defects while
others may {\em subtract} (and the final defect count is the number of additions, less the
number of subtractions). 

Two other models that use the variables of \fig{cocont} are the COCOMO {\em months} and {\em risk} model.
The {\em months} model predicts for total development time and   can be used to determine staffing levels
for a software project. For example, if {\em effort}=200  
and {\em months}=10, then this project needs 
$\frac{200}{10} =20$
developers.

As to the {\em risk} model, certain management decisions decrease the 
odds of successfully completing a project. For example suppose a manager demands
{\em more}  reliability ({\em rely}) while  {\em decreasing} analyst capability ({\em acap}).
Such a project is ``risky'' since it means the manager is demanding more reliability from less skilled analysts.
The COCOMO {\em risk} model contains dozens of rules that trigger on each
such ``risky'' combinations of decisions. 

XOMO is a challenging optimization problem.
It is difficult to reduce all of {\em months}, {\em effort}, {\em defects} and {\em risk} because they are conflicting objectives: 
some decisions that reduce one objective can increase another.
For example,
XOMO contains many such scenarios where the objectives conflict; some examples are as follows:
\bi
\item
Increasing
software reliability   {\em reduces} the
  number of added defects while {\em increasing} the 
software development effort;
\item 
Better documentation can improve team communication and {\em decrease} the number of introduced defects.
However, such increased documentation {\em increases} the development effort.
\ei
Prior work with XOMO~\cite{me09a} found that different optimizations are found if we explore (1)~the entire
XOMO input space or (2)~just the inputs relevant to a particular project. Put another way: what works best for one case may not work best for another case.
Hence, we run XOMO for the three different specific cases shown in \fig{xomocases}.

Each of these cases are software projects that were specified by domain experts from the NASA Jet Propulsion Laboratory (JPL).   In \fig{xomocases}, ``fl''
is a general description of all JPL flight software while ``o2''
describes version two of the flight guidance system
of the Orbital Space Plane.

Note that some of the COCOMO variables
range from some {\em low} to {\em high} value while others have a fixed {\em setting}.
For example, for ``o2'', reliability is fixed to {\em rely=5}, which is
its highest possible value.  
